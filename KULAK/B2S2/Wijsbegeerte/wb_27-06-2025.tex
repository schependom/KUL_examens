\documentclass[kulak]{kulakarticle}

\usepackage[dutch]{babel}
\usepackage{subcaption}
\usepackage{parskip}


\newcommand{\rood}[1]{\textcolor{red}{#1}}

% SI UNITX
\usepackage{siunitx}
\sisetup{
	per-mode=fraction,
	detect-weight = true,
	detect-family = true,
	separate-uncertainty=true, % Dit is voor de plus-minus
	output-decimal-marker={,}
}

\DeclareSIUnit\hour{uur}

\usepackage{bm}
\usepackage{tikz}
\usepackage{amsmath}
\usepackage{amsfonts}
\usepackage{mathtools}
\usepackage{empheq}

\title{Examen Wijsbegeerte}
\author{Vincent Van Schependom}
\date{27 juni 2025}
\address{
	\textbf{Groep Wetenschap \& Technologie Kulak} \\
	2\textsuperscript{e} Bach. Informatica: Wijsbegeerte \\
	Dr. Hanna Vandenbussche}

\begin{document}

	\maketitle

	\section*{Open vragen (10 pt.)}

	\subsection*{Vraag 1}

	Leg uit hoe de visie van Thomas Kuhn met betrekking tot wetenschap verschilt van die van Carl Popper.

	\subsection*{Vraag 2}

	Kant beweerde dat er nooit een Newton van de biologie zou komen. Een tijdje later publiceerde Darwin echter zijn evolutietheorie.
 	Leg uit 1) wat de twee principes achter de evolutietheorie van Darwin zijn en 2) welke invloed dit had op het teleologisch beeld van de natuur.

	\subsection*{Vraag 3}

	Geef 2 argumenten tegen verbetering die tijdens de hoorcolleges en in het handboek werden besproken. Geef ook zelfgekozen voorbeelden om jouw argumentatie te verduidelijken.

	\section*{MPC (10 pt.)}

	Er is maar één antwoord mogelijk. Er is giscorrectie (per fout antwoord wordt 1/3 punt afgetrokken).

	\begin{enumerate}
		\item Wat is aptonemophilia?
		\begin{enumerate}
			\item De ziekte waardoor je denkt dat je dood bent
			\item De drang om een gezond lichaamsdeel te amputeren
			\item ...
			\item De ziekte waarbij je verlegen bent
		\end{enumerate}
		\item In mei 2026 worden de Enhanced Games georganiseerd (in Las Vegas). Welke stelling is het meest van toepassing bij dit sportevenement?
		\begin{enumerate}
			\item Er bestaat geen grens tussen therapie en verbetering.
			\item De mens kan biomedische technologie gebruiken om zichzelf gezonder te maken.
			\item De mens kan biomedische technologie gebruiken om zijn lichaam te manipuleren en zo zijn prestaties te verbeteren.
			\item ...
		\end{enumerate}
		\item Wat betekent corroboreren?
		\begin{enumerate}
			\item Een hypothese vervangen door een andere hypothese
			\item Bevestigen
			\item Verifiëren
			\item Falsifiëren
		\end{enumerate}
		\newpage
		\item Wat is een argument om obesitas niet te pathologiseren?
		\begin{enumerate}
			\item Er is geen data die zegt dat obesitas per se een risico is voor andere?? hoezo andere?? diabetes of lage bloeddruk.
			\item Intussen spreekt men van \textit{globesatias}. Het is dus een probleem van het mens-zijn.
			\item De ziektevoorwaarden zijn niet voldaan. Bij bepaalde vormen van zwaarlijvigheid is er geen sprake van een disfunctie.
			\item Het is altijd een gevolg van te veel eten, dus het is geen ziekte.
		\end{enumerate}
		\item Waar wordt gebruik gemaakt van `prototype resemblance analysis'?
		\begin{enumerate}
			\item Pluralisme
			\item Naturalisme
			\item Normativisme
			\item Hybridisme
		\end{enumerate}
		\item Welke 2 (soms tegenstrijdige) stellingen hanteerden antieke filosofen\footnote{Er wordt verwezen naar Plato en Epicurus (stond niet op het examen)} met betrekking tot de dood?
		\begin{enumerate}
			\item Het is de bevrijding van de geest uit de kerker van het lichaam / Nadenken over de dood is een voorwaarde goed om te leven
			\item Het is de bevrijding van de geest uit de kerker van het lichaam / Het is een neutraal einde van het leven
			\item Het is een neutraal einde van het leven / ...
			\item .../...
		\end{enumerate}
		\item Bij welke opvatting over het menselijk lichaam sluit de eend van Vaucanson het aan?
		\begin{enumerate}
			\item Het lichaam is zoals een bezielde machine
			\item Het lichaam functioneert zoals een machine
			\item Het lichaam werkt doelgericht
			\item Het lichaam is kosmologisch
		\end{enumerate}
		\item Wat past het best bij de  `malin genie' en bij de filosofie van Decartes?
		\begin{enumerate}
			\item ...
			\item We kunnen twijfelen aan alles behalve aan het feit dat we twijfelen
			\item Onze zintuigen bestaan misschien niet
			\item ...
		\end{enumerate}
		\item Wat is geldige kritiek op `the argument from design'?
		\begin{enumerate}
			\item Het vertrekt vanuit de veronderstelling van een doelgerichte natuur
			\item De tweede premisse (= design vereist een designer) leidt tot een oneindige regressie.
			\item De eerste premisse (= er is design) leidt tot een oneindige regressie.
			\item ...
		\end{enumerate}
		\item Hoeveel vragen van de MPC hebben we onthouden?
		\begin{enumerate}
			\item Alle 10 de vragen.
			\item Slechts 9 vragen.
		\end{enumerate}
	\end{enumerate}



\end{document}
