\documentclass[kulak]{kulakarticle}

\usepackage[dutch]{babel}
\usepackage{subcaption}
\usepackage{parskip}


\newcommand{\rood}[1]{\textcolor{red}{#1}}

% SI UNITX
\usepackage{siunitx}
\sisetup{
	per-mode=fraction,
	detect-weight = true,
	detect-family = true,
	separate-uncertainty=true, % Dit is voor de plus-minus
	output-decimal-marker={,}
}

\DeclareSIUnit\hour{uur}

\usepackage{bm}
\usepackage{tikz}
\usepackage{amsmath}
\usepackage{amsfonts}
\usepackage{mathtools}
\usepackage{empheq}

\title{Examen Declaratieve Talen}
\author{Vincent Van Schependom}
\date{20 juni 2025}
\address{
	\textbf{Groep Wetenschap \& Technologie Kulak} \\
	2\textsuperscript{e} Bach. Informatica: Declaratieve Talen \\
	Patrick De Causmaecker, Pieter Bonte}

\begin{document}

	\maketitle

	Het examen duurt 3 uur. Toegelaten documentatie is \href{https://www.swi-prolog.org}{www.swi-prolog.org}, \href{https://hoogle.haskell.org}{hoogle.haskell.org}, samen met de
	cursus en de slides. Zend op het einde je antwoorden naar \href{mailto:patrick.decausmaecker@kuleuven.be}{patrick.decausmaecker@kuleuven.be} en \href{mailto:pieter.bonte@kuleuven.be}{pieter.bonte@kuleuven.be}.

	\section{Prolog \((\frac{5}{14})\)}

	We willen een predicaat \texttt{star(+Commands)} dat, gegeven een lijst van commando’s, sterretjes op een bord print. Er zijn 5 commando’s: \texttt{up}, \texttt{down}, \texttt{left}, \texttt{right} en \texttt{star}.

	Er is een probleem, want de linkerbovenhoek van het bord moet altijd in de eerste kolom staan. Dit wil zeggen dat de volgende commando's allemaal hetzelfde printen naar het scherm:

	\begin{verbatim}
?- star([right,right,right,right,right,star]).
*
?- star([left,left,left,left,left,star]).
*
?- star([down,up,left,right,left,left,star]).
*\end{verbatim}

	\subsection{}

	Schrijf het predicaat \texttt{collect\_stars(+Commandos, -Sterren)} dat, gegeven een lijst van commando's, een lijst van sterren genereert.
	\begin{verbatim}
?- collect_stars([right,right,right,right,right,star],S).
S = [star(5,0)]

?- collect_stars([down,star,left,down,right,up,up,star,right],S).
S = [star(0,-1),star(0,0)]\end{verbatim}
	\textit{Tip}: bij deze eerste implementatie moet je er nog geen rekening mee houden dat we moeten starten op positie 0.

	\subsection{}
	Schrijf het predicaat \texttt{write1line(+StarList, +Y, +StartCol)} die één lijn sterren afdrukt uit \texttt{StarList}. Hierbij is \texttt{Y} de coördinaat van de rij die we willen afdrukken en \texttt{StartCol} de kolom waar het bord moet starten.

	\begin{verbatim}
?- write1line([star(0,0),star(1,1),star(2,0)],0,0).
* *

?- write1line([star(0,0),star(1,1),star(2,0)],0,-5).
* *\end{verbatim}

	\newpage
	\subsection{}
	Vervolledig nu je implementatie door het predicaat \texttt{star/1} te schrijven dat de initialen print van je favoriete vak:

	\begin{verbatim}
?- star([star,up,star,up,star,up,star,up,right,right,right,star,left,star,left,
star,left,star,left,star,left,star,left,star,left,left,left,left,left,down,star,
down,star,down,star,down,left,star,left,star,left,star,left,star,up,star,up,star,
up,star,up,star,right,star,right,star,right,star]).

****     *******
*   *       *
*   *       *
*   *       *
****        *\end{verbatim}

	\section{Haskell \((\frac{5}{14})\)}

	\subsection{Veeltermen hebben nulpunten en dus coëfficiënten}

	In deze oefening maken we gebruik van \texttt{Double}, en van \texttt{Complex} uit \texttt{Data.Complex}.

	Veeltermen hebben een lijst van coëfficiënten en worden voorgesteld door \textbf{Polynoom}. Een lijst van nulpunten wordt voorgesteld door \textbf{Nullen}. Schrijf een functie \texttt{polynoom} die, gegeven een aantal nulpunten, de coëfficiënten berekent van de veelterm met 1 als hoogstegraadsterm.

	\begin{verbatim}
ghci> polynoom (N [1])
P [1.0,-1.0]

ghci> polynoom (N [2,3])
P [1.0,-5.0,6.0]

ghci> polynoom (N [2,3,4])
P [1.0,-8.0,26.0,24.0]

ghci> polynoom (N [1.0 :+ 2.0])
P [1.0 :+ 0.0, -1.0 :+ 2.0]\end{verbatim}

	\textit{Hint}: doe eens \((x - a)(x - b)(x - c)\). (Deze hint werd niet gegeven op het examen zelf)

	\subsection{De voorstelling kan beter}
	\footnote{Deze oefening is facultatief indien je 2.3.2 maakt} De voorstelling van de polynomen hierboven was eerder arbitrair. Schrijf een functie show die veeltermen duidelijker afprint.

	\begin{verbatim}
ghci> polynoom (N [1.0,2.0])
+ 1.0x^2 -3.0x + 2.0

ghci> polynoom (N [2,3])
+ 1.0x^2 - 5.0x + 6.0

ghci> polynoom (N [2,3,4])
+ 1.0x^3 - 8.0x^2 + 26.0x + 24.0

ghci> polynoom (N [1.0 :+ 2.0])
+ 1.0x - (1.0+2.0i)\end{verbatim}

	\subsection{De \texttt{State} Monad}

	\subsubsection{Random getallen}
	\footnote{Deze oefening is facultatief indien je 2.3.2 maakt} Schrijf een functie \texttt{randomAdd} die een random getal toevoegt aan een lijst. Gebruik de implementatie in bijlage voor het gebruik van de \texttt{State} Monad, alsook de \texttt{randomList} methode en de \texttt{RandomGen} typeclass die gezien werden in de les.

\begin{verbatim}
ghci> radd4 = randomAdd [] >>= randomAdd >>= randomAdd >>= randomAdd

ghci> fst $ runState radd4 (R 17)
[185,89,81,17]

ghci> snd $ runState radd4 (R 17)
R 3145170771490243124\end{verbatim}

	Toon de wet van associativiteit van Monads aan door \texttt{radd4assoc} te definiëren waarbij deze wet wordt gebruikt.

	\subsubsection{Vergeet niet wat werd gedaan}

	Schrijf functies \texttt{livingListPlus} en \texttt{livingListMin} die een \texttt{State} teruggeven waarbij een element ofwel aan een lijst kan worden toegevoegd ofwel verwijderd kan worden uit die lijst.

	\begin{verbatim}
ghci> runState (livingListPlus 10 [1,2,3]) []
([10,1,2,3],[T 10])

ghci> runState (livingListMin 10 [1,2,3,10]) []
([1,2,3],[M 10])

ghci> runState (livingListMin 10 [1,2,3]) []
([1,2,3],[])

ghci> runState (livingListMin 2 [1,2,3,2,3,1,2,1]) []
([1,3,3,1,1],[M 2,M 2,M 2])

\end{verbatim}

	Gebruik bovenstaande functies om een functie \texttt{doeDeKlinkersWeg} te schrijven die alle klinkers uit een gegeven string kan verwijderen.

\begin{verbatim}
ghci> fst $ runState (doeDeKlinkersWeg "Dit willen we vandaag over Haskell horen.") []
"Dt wlln w vndg vr Hskll hrn."

ghci> snd $ runState (doeDeKlinkersWeg "Dit willen we vandaag over Haskell horen.") []
[M 'o',M 'o',M 'i',M 'i',M 'e',M 'e',M 'e',M 'e',M 'e',M 'a',M 'a',M 'a',M 'a']\end{verbatim}

	\newpage
	\section{Theorie \((\frac{4}{14})\)}

	\begin{enumerate}
		\item Wat doen de functies \texttt{p1/2}, \texttt{p2/2} en \texttt{p3/3}? Leg het verschil in uitvoering uit.
\begin{verbatim}
p1([], []).
p1([X|Xs], R) :-
		p1(Xs, Rs),
		append(Rs, [X],R).

p2(L, R) :-
		p2_2(L, R-[]).
p2_2([], X-X).
p2_2([H|T], R-X) :-
		p2_2(T, R-[H|X]).

p3(L, R) :-
		p3_2(L, [], R).
p3_2([], R, R).
p3_2([H|T], L, R) :-
		p3_2(T, [H|L], R).
\end{verbatim}
		\item Bewijs dat voor de Monad\(^\dagger\) instantie van \texttt{Tree} geldt dat \(join \circ join \equiv join \circ fmap \, join\).
		\item \footnote{Onderscheidingsvraag} Wat is het verschil in uitvoeringssnelheid tussen onderstaande reeksen commando’s in Haskell?

		Reeks 1:
		\begin{verbatim}
ghci> fst $ runState rl (R 17)
[17,81,89,185,52,48,86]
ghci> snd $ runState rl (R 17)
R 5741632773585689292\end{verbatim}

		Reeks 2:
		\begin{verbatim}
ghci> runState rl (R 17)
([17,81,89,185,52,48,86],R 5741632773585689292)\end{verbatim}
	\end{enumerate}





\newpage
	\section*{Bijlage}

	\begin{verbatim}
import Control.Monad
import Data.Bits
import Data.Int

-- === Code paragraaf "7.4.3 The State Monad" in "bookHaskellTS.pdf" ===

type SP s a = s -> (a, s)

data State s a = State (SP s a)

bindSP :: SP s a -> (a -> SP s b) -> SP s b
bindSP m f = \s0 ->
	( let (x, s1) = m s0
				in f x s1
	)

pureSP :: a -> SP s a
pureSP x = \s -> (x, s)

instance Functor (State s) where
fmap = liftM

instance Applicative (State s) where
		pure x = State (pureSP x)
		(<*>) = ap

instance Monad (State s) where
		m >>= f = State (bindSP (runState m) (runState . f))

runState :: State s a -> SP s a
runState (State m) = m

get :: State s s
get = State (\s -> (s, s))

put :: s -> State s ()
put s' = State (\s -> ((), s'))

modify :: (s -> (a, s)) -> State s a
modify f = State f

-- =======================================================
-- Eigen implementatie van een random-generator
-- zie System.Random voor een Haskell standaard

class RandomGen g where
next :: g -> (Int, g)

-- random number generator (Marsaglia, George (July 2003). "Xorshift RNGs")
xsl x v = x `xor` (shiftL x v)

xsr x v = x `xor` (shiftR x v)

xorShift x = xsl (xsr (xsl x 13) 7) 17

data R = R Int deriving (Show)

instance RandomGen R where
next (R x) = (mod x 256, R (xorShift x))

-- --------------------------------------
randomList :: (RandomGen g) => Integer -> State g [Int]
randomList n
		| n > 0 = do
					x <- modify next
					xs <- randomList (n - 1)
					return (x : xs)
		| otherwise = return []

{-
-- voorbeeld: rl is een State
-- runState rl (R 17) laat deze lopen met initiele waarde 17 voor de randomgenerator
-- het resultaat is een koppel bestaande uit de lijst van random getallen en de nieuwe
-- toestand van de randomgenerator.

ghci> rl = randomList 7
ghci> fst $ runState rl (R 17)
[17,81,89,185,52,48,86]
ghci> snd $ runState rl (R 17)
R 5741632773585689292
ghci>
-}
	\end{verbatim}

\end{document}
