\documentclass[kulak]{kulakarticle} % options: kulak (default) or kul

\usepackage[dutch]{babel}
\usepackage{amssymb, amsthm, amsmath}
\usepackage{siunitx}
\usepackage{graphicx}

\title{Examen Wiskundig Redeneren}
\author{Vincent Van Schependom}
\date{12 januari 2024}
\address{\textbf{Wiskundig Redeneren}\\
	Prof. Karel Dekimpe}

\begin{document}

	\maketitle

	\section*{Vraag 1}

	Zijn onderstaande beweringen waar of onwaar? Verklaar waarom.

	\begin{enumerate}
		\item Zij \(A\) een eindige verzameling. De kardinaliteit van de verzameling van alle relaties van \(A\) naar zichzelf is gelijk aan \(2^{(\#A)^2}\).
		\item De GGF ... is een tautologie.
		\item Beschouw de afbeeldingen \(f:X\rightarrow Y\) en \(g:Y \rightarrow X\). Als \(f\circ g \circ f\) een bijectie is, is \(f\) dat ook.
	\end{enumerate}

	\section*{Vraag 2}

	Beschouw de afbeelding \(f:\mathbb{N}\rightarrow\mathbb{R}:n\mapsto\sqrt{n}\) en de verzamelingen
	\begin{itemize}
		\item[-] \(X_0=\{0\}\)
		\item[-] \(X_1=\{1\}\)
		\item[-] \(X_n=f^{-1}([0,n])\setminus X_{n-1}\) \hspace{1cm} voor \(n\geq2\) met \(n\in \mathbb{N}_0\)
	\end{itemize}

	\noindent Gevraagd:

	\begin{enumerate}
		\item Geef \(X_3\).
		\item Bewijs dat voor elke \(n \in \mathbb{N}\) geldt dat \[\#X_n=\frac{(-1)^n+n^2+n+1}{2}\]
		\item Bewijs dat voor elke \(n \in \mathbb{N}\) geldt dat \(X_{n}\subset X_{n+2}\)
	\end{enumerate}

	\section*{Vraag 3}

	Een oefening op de - \textit{partieel (en dus niet totaal), maar wel volledige} - geordende verzameling \((\mathcal{P}(X),\subset)\) en infimum. Lijkt heel hard op oefening 8b.

	\section*{Vraag 4}

	\begin{enumerate}
		\item Zij \(R\) een relatie. Bewijs dat \(R\) symmetrisch is a.s.a. \(R^{-1}\subset R\)
		\item Beschouw twee relaties \(R\) en \(S\) met \(S=R\setminus R^{-1}\). Bewijs of geef een tegenvoorbeeld:
		\begin{enumerate}
			\item \(S\) is reflexief
			\item \(S\) is symmetrisch
			\item \(S\) is anti-symmetrisch
			\item \(S\) is transitief
		\end{enumerate}
	\end{enumerate}

	\newpage

	\section*{Vraag 5}

	Tuur is op bezoek bij opa. In zijn speeldoos zitten speelgoedauto's van verschillende kleuren: 3 blauwe, 4 gele, 6 rode en 7 witte.

	\begin{enumerate}
		\item Hoeveel verschillende combinaties zijn er mogelijk indien alle auto's van dezelfde kleur naast elkaar moeten staan?\\
		\textit{Ga er hierbij vanuit dat twee auto's van dezelfde kleur onderling verschillend zijn.}
		\item Tuur moet 10 auto's uit de doos halen en minstens 2 daarvan moeten wit zijn.
		\item Hoeveel verschillende kleurencombinaties zijn er mogelijk indien je maar 4 auto's mag gebruiken?
		\textit{Ga er hierbij vanuit dat twee auto's van dezelfde kleur gelijkwaardig zijn.}
	\end{enumerate}


\end{document}
