\documentclass[kulak]{kulakarticle} % options: kulak (default) or kul

\usepackage[dutch]{babel}
\usepackage{amssymb, amsthm, amsmath}
\usepackage{siunitx}
\usepackage{graphicx}
\usepackage{amsthm}

\DeclareMathOperator{\Bgsin}{Bgsin}

\setlength{\parindent}{0pt}

\newcommand{\R}{\mathbb{R}}

\title{Examen Analyse \& Calculus}
\author{Vincent Van Schependom}
\date{23 januari 2024}
\address{
	\textbf{Analyse \& Calculus I}\\
	Prof. Wim Malfait}

\begin{document}

	\maketitle

	\section*{Open vragen}

	\subsection*{Vraag 1}
	\renewcommand{\arraystretch}{2}
	Beschouw de functie \[ f :\R \to \R : x \mapsto \left\{
	\begin{array}{ll}
		\dfrac{  e^{-1/x^2}  }{ |x| } 	& \text{als } x\neq0 \\
		0 														& \text{als } x=0
	\end{array} \right. \ \]

	\begin{enumerate}
		\item Toon aan dat \(f\) afleidbaar is in \(0\).
		\item Bepaal de afgeleide functie \(f'\).
		\item Onderzoek en bespreek de extreme waarden van \(f\).
	\end{enumerate}

	\subsection*{Vraag 2}

	Beschouw de functie \(f\) en het gebied \(D\): \[ f : \R \to \R : (x,y,z) \mapsto f(x,y,z) = yz \]\[ D = \{ (x,y,z) \in \R^3 \mid x^2+y^2=18 \text{ en } z=2x \} \]

	In welk(e) punt(en) bereikt \(f\) haar minimum?

	\subsection*{Vraag 3}

	Beschouw het lichaam \(D\) dat ingesloten is tussen de oppervlakken \(x^2=y^2+z^2\) en \(x^2+4y^2+4z^2=4\) in de halfruimte \(x\geq0\).

	\begin{enumerate}
		\item Beschrijf \(D\) door middel van (gepaste) cilindercoördinaten.
		\item Bereken het volume van \(D\).
	\end{enumerate}

	\section*{Meerkeuzevragen}

	\subsection*{Vraag 1}

	De booglengte van de kromme met vergelijking \[ y=\sqrt{x-x^2} + \Bgsin \sqrt{x} \] voor \(x\in[0,1]\) is gelijk aan
	\begin{itemize}
		\item -1
		\item 0
		\item 2
		\item \(+\infty\)
	\end{itemize}

	\subsection*{Vraag 2}

	Beschouw de volgende reeksen:
	\begin{align*}
		\text{(I)} \sum_{n=1}^{+\infty}n\cdot\sin{\left(\frac{1}{n}\right)} && \text{(II)} \sum_{n=1}^{+\infty} \frac{1}{\ln^n(n+1)}
	\end{align*}

	\begin{itemize}
		\item (I) en (II) zijn beide divergent.
		\item (I) is divergent en (II) is convergent.
		\item (I) is convergent en (II) is divergent.
		\item (I) en (II) zijn beide convergent.
	\end{itemize}

	\subsection*{Vraag 3}

	Van welke functie is onderstaande machtreeks de reeksontwikkeling? \[ \sum_{n=1}^{+\infty}\frac{2^n}{n}x^{2n} \]

	\begin{itemize}
		\item \( -\ln{(1-2x^2)} \)
		\item \( -\ln{(1+2x^2)} \)
		\item \( \ln{(1-2x^2)} \)
		\item \( \ln{(1+2x^2)} \)
	\end{itemize}

	\subsection*{Vraag 4}

	Beschouw de functie \[ f : \R^2 \to \R : (x,y) \mapsto -e^{-\sqrt{x^2+y^2}} \]
	Welke van onderstaande grafieken met niveaulijnen hoort bij de functie \(f\)? Dit was echt ontiegelijk makkelijk aangezien de grafiek van \(f\) in 3D ook gewoon gegeven was.

	\subsection*{Vraag 5}

	\(y\) is de oplossing van de differentiaalvergelijking \[ y' + (\sin{x})y = e^{\cos{x}} \] met beginvoorwaarde \(y(0)=0\). \(y(\pi)\) is gelijk aan
	\begin{itemize}
		\item \(0\)
		\item \(\frac{1}{e}\)
		\item \(\frac{\pi}{e}\)
		\item \(\pi\)
	\end{itemize}

\end{document}
