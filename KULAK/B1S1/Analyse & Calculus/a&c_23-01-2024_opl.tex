\documentclass[kulak]{kulakarticle} % options: kulak (default) or kul

\usepackage[dutch]{babel}
\usepackage{amssymb, amsthm, amsmath}
\usepackage{siunitx}
\usepackage{graphicx}
\usepackage{amsthm}

\DeclareMathOperator{\Bgsin}{Bgsin}

\setlength{\parindent}{0pt}

\newcommand{\R}{\mathbb{R}}

\renewcommand{\boxed}[1]{\text{\fboxsep=.3em\fbox{#1}}}

\newcommand{\llim}{\underset{<}{\lim_{x\to0} \,} }
\newcommand{\rlim}{\underset{>}{\lim_{x\to0} \,} }
\newcommand{\hopital}{\overset{\mathrm{H}}{=}}

\title{Oplossing Examen Analyse \& Calculus}
\author{Vincent Van Schependom}
\date{23 januari 2024}
\address{
	\textbf{Analyse \& Calculus I}\\
	Wim Malfait}
\begin{document}

	\maketitle

	\section*{Open vragen}

	\subsection*{Vraag 1}
	\renewcommand{\arraystretch}{2}
	Beschouw de functie \[ f :\R \to \R : x \mapsto \left\{
	\begin{array}{ll}
		\dfrac{  e^{-1/x^2}  }{ |x| } 	& \text{als } x\neq0 \\
		0 														& \text{als } x=0
	\end{array} \right. \ \]

	\begin{enumerate}
		\item Toon aan dat \(f\) afleidbaar is in \(0\).
		\item Bepaal de afgeleide functie \(f'\).
		\item Onderzoek en bespreek de extreme waarden van \(f\).
	\end{enumerate}

	\textbf{\underline{Oplossing}}:\\

	We berekenen de rechterafgeleide in \(x=0\) m.b.v. de definitie van afgeleiden:
	\begin{align*}
		\rlim \frac{f(x)-f(0)}{x-0} & = \rlim \frac{\frac{  e^{-1/x^2}  }{ |x| } }{x} \\
													& = \rlim \frac{\frac{  e^{-1/x^2}  }{ x } }{x} \\
													& = \rlim \frac{  e^{-1/x^2}  }{x^2} \\
													& = \rlim \frac{  \frac{1}{x^2}  }{e^{1/x^2}} \\
													& \hopital \rlim \frac{  (-2x^{-3})  }{e^{1/x^2} \cdot (-2x^{-3}) } \\
													& = \rlim  \frac{  1  }{e^{1/x^2} } \\
													& = \boxed{0} \\
	\end{align*}
	Analoog berekenen we de linkerafgeleide:
	\begin{align*}
		\llim \frac{f(x)-f(0)}{x-0} & = \llim \frac{\frac{  e^{-1/x^2}  }{ |x| } }{x} \\
													& = \llim \frac{\frac{  e^{-1/x^2}  }{ -x } }{x} \\
													& = \boxed{0}
	\end{align*}

	Omdat \(0\in \text{def}(f)\) en \(f_l'(0) = f_r'(0) = 0\), is \(f\) afleidbaar in 0 en is \(f'(0)=0\).\newpage

	\[ f :\R \to \R : x \mapsto \left\{
	\begin{array}{ll}
		\dfrac{  e^{-1/x^2}  }{ -x } 	& \text{als } x<0 \\
		0 												& \text{als } x=0 \\
		\dfrac{  e^{-1/x^2}  }{ x } 	& \text{als } x>0 \\
	\end{array} \right. \ \]

	Voor \(x<0\):
	\begin{align*}
		f_{1}'(x) = \frac{d}{dx} \left( \frac{e^{-1/x^2}  }{ -x }  \right) = \frac{ (-x) \cdot e^{-1/x^2} \cdot (2x^{-3}) - e^{-1/x^2} \cdot (-1) }{x^2} = e^{-1/x^2}(x^{-2}-2x^{-4})
	\end{align*}

	Voor \(x>0\):
	\begin{align*}
		f_{2}'(x) = \frac{d}{dx} \left( \frac{e^{-1/x^2}  }{ x }  \right) = \frac{ x \cdot e^{-1/x^2} \cdot (2x^{-3}) - e^{-1/x^2} }{x^2} = e^{-1/x^2}(2x^{-4}-x^{-2})
	\end{align*}

	En dus wordt de afgeleide functie \(f'\) van \(f\): \[
	f' :\R \to \R : x \mapsto \left\{
	\begin{array}{ll}
		e^{-1/x^2}(x^{-2}-2x^{-4})	& \text{als } x<0 \\
		0 												& \text{als } x=0 \\
		e^{-1/x^2}(2x^{-4}-x^{-2})	& \text{als } x>0 \\
	\end{array} \right. \ \]

	Om de kandidaat-extrema te bepalen, berekenen we de nulpunten van \(f_1'\) en \(f_2'\):
	\begin{align*}
		f'(x)=0 & \Leftrightarrow (f_{1}'(x)=0  \land x<0) \lor (f_{2}'(x)=0 \land x>0) \\
					& \Leftrightarrow \boxed{\(x=\sqrt{2} \lor x=-\sqrt{2}\)}
	\end{align*}

	Maak een tekenschema. Je zal zien dat \(f_1'\) in \(x=-\sqrt{2}\) wisselt van teken (van positief naar negatief) en dat \(f_2'\) van teken wisselt in \(x=\sqrt{2}\) (ook van positief naar negatief). We besluiten dat \(f\) in beide punten een relatief maximum bereikt.\\

	Aangezien nu ook geldt dat \(\lim_{x\to +\infty} \, f(x) = \lim_{x\to -\infty} \, f(x) =0\), kunnen we concluderen dat \(f\) zowel voor \(x=\sqrt{2}\) als voor \(x=-\sqrt{2}\) een \textbf{absoluut} maximum bereikt.

	\newpage

	\subsection*{Vraag 2}

	Beschouw de functie \(f\) en het gebied \(D\): \[ f : \R \to \R : (x,y,z) \mapsto f(x,y,z) = yz \]\[ D = \{ (x,y,z) \in \R^3 \mid x^2+y^2=18 \text{ en } z=2x \} \]

	In welk(e) punt(en) bereikt \(f\) haar minimum? \\

	\textbf{\underline{Oplossing}}: \\

	We stellen de Lagrangefunctie op: \[ L : \R^5\to\R : (x,y,z,\lambda,\mu) \mapsto yz + \lambda x^2 +\lambda y^2 - 18 \lambda + \mu z - 2 \mu x \]

	We berekenen de kritieke punten:

	\renewcommand{\arraystretch}{2}
	\[ \left\{
	\begin{array}{l}
		0 = \dfrac{\partial L}{\partial x} = 2x \lambda -2\mu \\
		0 = \dfrac{\partial L}{\partial y} = z + 2\mu y \\
		0 = \dfrac{\partial L}{\partial z} = y + \mu \\
		0 = \dfrac{\partial L}{\partial \lambda} = x^2+y^2-18 \\
		0 = \dfrac{\partial L}{\partial \mu} = z - 2x \\
	\end{array}
	\right. \]

	Na het oplossen van dit stelsel vinden we 4 kandidaat-extrema. We berekenen de functiewaarden van elk punt:
	\begin{itemize}
		\item \(f(-3,-3,-6)=18\)
		\item \(f(-3,3,-6)=-18\)
		\item \(f(3,-3,6)=-18\)
		\item \(f(3,3,6)=18\)
	\end{itemize}

	We vinden dus twee minima voor \(f\) op \(D\), namelijk \boxed{\((-3,3,-6)\) en \((3,-3,6)\)}.

	\newpage

	\subsection*{Vraag 3}

	Beschouw het lichaam \(D\) dat ingesloten is tussen de oppervlakken \(x^2=y^2+z^2\) en \(x^2+4y^2+4z^2=4\) in de halfruimte \(x\geq0\).

	\begin{enumerate}
		\item Beschrijf \(D\) door middel van (gepaste) cilindercoördinaten.
		\item Bereken het volume van \(D\).
	\end{enumerate}

	\textbf{\underline{Oplossing}}: \\

	We berekenen eerst de doorsnede van beide oppervlakken.
	\begin{align*}
		x^2+4y^2+4z^2 &=4 \\
		& \Updownarrow x^2 = y^2+z^2 \\
		(y^2+z^2) + 4y^2 + 4z^2 &= 4 \\
		& \Updownarrow \\
		5y^2+5z^2&=2^2 \\
		& \Updownarrow \\
		y^2 + z^2 &= \left( \frac{2}{\sqrt{5}} \right)^2
	\end{align*}

	We kunnen \(D\) op volgende manier beschrijven m.b.v. cilindercoördinaten:
	\renewcommand{\arraystretch}{1}
	\begin{align*}
	\left\{
	\begin{array}{l}
		y = r \cos(\theta)\\
		z = r \sin(\theta)\\
		x = x
	\end{array}
	\right. & & |\mathrm{J}| = r
	\end{align*}

	\[D = \left\{ (r,\theta,x) \in \R^{+} \times [0,2\pi[ \, \times \, \R \mid 0 \leq r \leq \frac{2}{\sqrt{5}} \text{ en } 0 \leq \theta \leq 2\pi \text{ en } r \leq x \leq 2\sqrt{1-r^2} \right\}\]

	Het volume van \(D\) wordt dan:
	\begin{align*}
		\iiint\limits_D r\;dx \, dr \, d\theta &= \int_0^{2\pi} \left( \int_0^{2/\sqrt{5}} \left( \int_{r}^{2\sqrt{1-r^2}} r \, dx \right) dr \right) d\theta \\
		& = \boxed{\(\frac{20-4\sqrt{5}}{15}\pi\)} \\
		& \approx 2.3155
	\end{align*}

	\newpage

	\section*{Meerkeuzevragen}

	\subsection*{Vraag 1}

	De booglengte van de kromme met vergelijking \[ y=\sqrt{x-x^2} + \Bgsin \sqrt{x} \] voor \(x\in[0,1]\) is gelijk aan
	\begin{itemize}
		\item -1
		\item 0
		\item \boxed{2}
		\item \(+\infty\)
	\end{itemize}

	\textbf{\underline{Oplossing}}: \\

	Stel \(f(x)=\sqrt{x-x^2} + \Bgsin \sqrt{x}\). De booglengte is dan gelijk aan \[L=\int_0^1 \sqrt{1+(f'(x))^2} \, dx\]
	We berekenen eerst \(f'(x)\):
	\begin{align*}
		\frac{d}{dx}\left(\sqrt{x-x^2} + \Bgsin \sqrt{x}\right)
		&= \frac{1}{2}\cdot(x-x^2)^{-\frac{1}{2}}\cdot(1-2x)+\frac{1}{\sqrt{1-x}}\cdot \left(\frac{1}{2}x^{-\frac{1}{2}}\right) \\
		&= \frac{1-2x}{2\cdot \sqrt{x-x^2}} + \frac{1}{2\cdot \sqrt{x}\cdot \sqrt{1-x}} \\
		&= \frac{2-2x}{2\cdot \sqrt{x}\cdot \sqrt{1-x}} \\
		&= \frac{1-x}{\sqrt{x}\cdot \sqrt{1-x}} \\
		&= \frac{\sqrt{1-x}}{\sqrt{x}}
	\end{align*}

	En dus wordt de onbepaalde integraal:
	\begin{align*}
		\int \sqrt{1+(f'(x))^2} \, dx
		&= \int \sqrt{1+\left(\frac{\sqrt{1-x}}{\sqrt{x}}\right)^2} \, dx \\
		&= \int \sqrt{1+\frac{1-x}{x}} \, dx \\
		&= \int \sqrt{\frac{x+1-x}{x}} \, dx \\
		&= \int \frac{1}{\sqrt{x}} \, dx \\
		&= \int x^{-\frac{1}{2}} \, dx \\
		&= 2x^{\frac{1}{2}} + c
	\end{align*}

	Tot slot berekenen we de booglengte:
	\begin{align*}
		L&=\int_0^1 \sqrt{1+(f'(x))^2} \, dx\\
		&= \left[2\sqrt{x}\right]^1_0\\
		&= 2
	\end{align*}

	\subsection*{Vraag 2}

	Beschouw de volgende reeksen:
	\begin{align*}
		\text{(I)} \sum_{n=1}^{+\infty}n\cdot\sin{\left(\frac{1}{n}\right)} && \text{(II)} \sum_{n=1}^{+\infty} \frac{1}{\ln^n(n+1)}
	\end{align*}

	\begin{itemize}
		\item (I) en (II) zijn beide divergent.
		\item \boxed{(I) is divergent en (II) is convergent.}
		\item (I) is convergent en (II) is divergent.
		\item (I) en (II) zijn beide convergent.
	\end{itemize}

	\textbf{\underline{Oplossing}}: \\

	Een nodige voorwaarde voor de convergentie van een reeks, is dat \(\lim_{n\to +\infty}a_n=0\). We kijken of dit het geval is voor (I):
	\begin{align*}
		\lim_{n\to +\infty} n\cdot \sin\left( \frac{1}{n} \right)
		&= \lim_{n\to +\infty} \frac{\sin\left( \frac{1}{n} \right)}{\frac{1}{n}}\\
		&\hopital \lim_{n\to +\infty} \frac{\cos\left( \frac{1}{n} \right)\cdot \left(-\frac{1}{n^2}\right)}{-\frac{1}{n^2}}\\
		&= \lim_{n\to +\infty} \cos\left( \frac{1}{n} \right)\\
		&=1
	\end{align*}

	Omdat voor (I) geldt dat \(\lim_{n\to +\infty}a_n\neq0\), besluiten we dat (I) \textbf{divergent} is.\\

	Voor (II) passen we de worteltest toe:
	\begin{align*}
		L
		&= \lim_{n\to +\infty} \sqrt[n]{\frac{1}{\ln^n(n+1)}}\\
		&= \lim_{n\to +\infty} \frac{1}{\ln(n+1)}\\
		&=0
	\end{align*}

	Omdat \(0\leq L \leq 1\), besluiten we dat (II) \textbf{convergent} is.

	\subsection*{Vraag 3}

	Van welke functie is onderstaande machtreeks de reeksontwikkeling? \[ \sum_{n=1}^{+\infty}\frac{2^n}{n}x^{2n} \]

	\begin{itemize}
		\item \boxed{\( -\ln{(1-2x^2)} \)}
		\item \( -\ln{(1+2x^2)} \)
		\item \( \ln{(1-2x^2)} \)
		\item \( \ln{(1+2x^2)} \)
	\end{itemize}

	\subsection*{Vraag 4}

	Beschouw de functie \[ f : \R^2 \to \R : (x,y) \mapsto -e^{-\sqrt{x^2+y^2}} \]
	Welke van onderstaande grafieken met niveaulijnen hoort bij de functie \(f\)? Dit was echt ontiegelijk makkelijk aangezien de grafiek van \(f\) in 3D ook gewoon gegeven was.

	\subsection*{Vraag 5}

	\(y\) is de oplossing van de differentiaalvergelijking \[ y' + (\sin{x})y = e^{\cos{x}} \] met beginvoorwaarde \(y(0)=0\). \(y(\pi)\) is gelijk aan
	\begin{itemize}
		\item \(0\)
		\item \(\dfrac{1}{e}\)
		\item \boxed{\(\dfrac{\pi}{e}\)}
		\item \(\pi\)
	\end{itemize}

\end{document}
