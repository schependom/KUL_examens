\documentclass[kulak]{kulakarticle} % options: kulak (default) or kul

\usepackage[dutch]{babel}

\usepackage{amsmath}
\usepackage{amssymb}
\usepackage{amsfonts} % for \mathbb
\usepackage{amsthm}
\usepackage{tcolorbox}
\usepackage{mathtools}


\newcommand{\R}{\mathbb{R}} % Real numbers
\newcommand{\C}{\mathbb{C}} % Complex numbers
\newcommand{\Q}{\mathbb{Q}}
\newcommand{\N}{\mathbb{N}}

\newcommand*\diff{\mathop{}\!\mathrm{d}}
\newcommand*\Diff[1]{\mathop{}\!\mathrm{d^#1}}

\title{Oplossing Examen Technieken voor Wiskundige Analyse (A\&C2)}
\author{Vincent Van Schependom}
\date{8 juni 2024}
\address{
	\textbf{Technieken voor Wiskundige Analyse} \\
	Prof. dr. Ir. Hans Dierckx\\\textit{(always remember)}}

\begin{document}

\maketitle

\section*{Vraag 1}

Gegeven is volgende functie:
\[f(x)=\begin{cases}
	1 & 0 \leq x \leq \frac{\pi}{2} \\
	0 & \frac{\pi}{2} < x < \pi
\end{cases} \]

\begin{enumerate}
	\item[a)] \(F(x)\) is de even periodieke uitbreiding van \(f\) met periode \(T=2\pi\). Maak een schets van \(F(x)\) voor \(-2\pi < x <2\pi\) en bepaal de Fourierreeks van deze functie.

	\underline{Oplossing:}
	Ik had een Fourierreeks met enkel cosinussen, waarbij ik een onderscheid maakte voor de mogelijke waarden van \(k\) voor de coëfficiënten \(a_k\):
	\begin{align*}
		 & k=0    &    &          & a_k&=1                   \\
		 & k=4n   & (n & \in\N_0) & a_k&=0                   \\
		 & k=4n+1 & (n & \in\N)   & a_k&=\frac{2}{(4n+1)\pi} \\
		 & k=4n+2 & (n & \in\N)   & a_k&=0                   \\
		 & k=4n+3 & (n & \in\N)   & a_k&=\frac{-2}{(4n+3)\pi}
	\end{align*}

	En dus wordt de Fourierreeks:
	\[g_1(x) = \frac{1}{2} + 2\sum_{k=0}^{\infty}\frac{1}{(4k+1)\pi}\cos\left[(4k+1)x\right] - 2\sum_{k=0}^{\infty}\frac{1}{(4k+3)\pi}\cos\left[(4k+3)x\right]\]

	\item[b)] \(G(x)\) is de oneven periodieke uitbreiding van \(f\) met periode \(T=2\pi\). Maak een schets van \(G(x)\) voor \(-2\pi < x <2\pi\) en bepaal de Fourierreeks van deze functie.

	\underline{Oplossing:}

	Hier had ik een gelijkaardige reeks, maar dan met enkel sinussen.

	\item[c)] Als de Fourierreeks van een functie \(g(x)\) gegeven wordt door
	\[ \frac{a_0}{2} + \sum_{k=1}^\infty{a_k\cos\left(\frac{2\pi kx}{T}\right)}
	+ \sum_{k=1}^\infty{b_k\sin\left(\frac{2\pi kx}{T}\right)}, \]
	dan geldt volgens de gelijkheid van Parseval dat
	\[ \frac{2}{T}\int_{-T/2}^{T/2}g^2(x)= \frac{a_0^2}{2} + \sum_{k=1}^\infty{(a_k^2+b_k^2)} \]

	Gebruik een van de hierboven berekende Fourierreeksen om de volgende som te berekenen:
	\[1+\frac{1}{3^2}+\frac{1}{5^2}+\frac{1}{7^2}+...\]

	\newpage

	\underline{Oplossing:}

	Omdat bij de Fourierreeks van \(F(x)\) de coëfficiënten \(a_k\) wegvallen voor even \(k\), en omdat bovendien geldt dat (reken na)
	\[\frac{2}{T}\int_{-T/2}^{T/2}F^2(x)=1,\]
	vinden we aan de hand van de gelijkheid van Parseval dat
	\begin{align*}
		1 & =\frac{a_0^2}{2} + \sum_{k=1}^\infty{(a_k^2+0^2)}   \\
		  & =\frac{1}{2}+\sum_{k=0}^{\infty}\left(\frac{2}{(4k+1)\pi}\right)^2+\sum_{k=0}^{\infty}\left(\frac{-2}{(4k+3)\pi}\right)^2\\
		  & =\frac{1}{2}+\sum_{k=0}^{\infty}\frac{4}{(2k+1)^2\pi^2}
	\end{align*}

	Hieruit volgt dat
	\begin{align*}
		\frac{1}{2} = 4\sum_{k=0}^{\infty}\frac{1}{(2k+1)^2\pi^2} \Longleftrightarrow \frac{\pi^2}{8} = \sum_{k=0}^{\infty}\frac{1}{(2k+1)^2}
	\end{align*}

	Besluit:
	\[1+\frac{1}{3^2}+\frac{1}{5^2}+\frac{1}{7^2}+... = \frac{\pi^2}{8}
	\]

\end{enumerate}

\section*{Vraag 2}

Een producent maakt blikken met een volume van 500 kubieke centimeter uit metaal. Hiervoor gebruiken ze ten eerste twee vierkanten, waarvan de zijde gelijk is aan de diameter van de schijf die eruit wordt gesneden voor het boven- en ondervlak. Daarnaast gebruiken ze voor de mantel van het blik een rechthoek met breedte \(h\). Omwille van esthetische redenen wil de producent dat de verhouding van de hoogte op de straal van het boven- en ondervlak (\(\frac{h}{r}\)) tussen de 3 en de 5 cm ligt.

\begin{enumerate}
	\item[a)] Bepaal de straal \(r\) en hoogte \(h\) waarvoor de producent zo weinig mogelijk metaal gebruikt.

	\underline{Oplossing:}

	Uit de voorwaarde i.v.m. het volume volgt dat
	\begin{align}
		r^2\pi h = 500 \Leftrightarrow \pi h = \frac{500}{r^2} \Leftrightarrow h = \frac{500}{r^2\pi} \label{vgl1}
	\end{align}

	De oppervlakte van het gebruikte metaal is
	\begin{align*}
		f(r,h) &= 2\cdot A_\text{vierkant boven- en ondervlak} + A_\text{rechthoek voor de mantel} \\
				&= 2\cdot (2r)^2 + h \cdot (2\pi r)\\
				&= 8r^2 + 2h\pi r
	\end{align*}

	Hierin kunnen we (\ref{vgl1}) substitueren:
	\begin{align*}
		\tilde{f}(r) &= 8r^2 + 2r \cdot \left(\frac{500}{r^2}\right)\\
					&= 8r^2 + 1000r^{-1}
	\end{align*}

	\newpage

	Uit de esthetische voorwaarde en uit (\ref{vgl1}) volgt dat
	\begin{align*}
		& 3 \leq \frac{h}{r} \leq 5 \\
		\Leftrightarrow\, & 3 \leq \frac{\left(\frac{500}{r^2\pi}\right)}{r} \leq 5 \\
		\Leftrightarrow\, & 3 \leq \frac{500}{r^3\pi} \leq 5 \\
		\Leftrightarrow\, & 3r^3\pi \leq 500 \leq 5r^3\pi
	\end{align*}

	Hieruit volgen twee nevenvoorwaarden (\(z,t \in \R\)):
	\begin{align*}
		\begin{cases}
			3r^3\pi - 500 \leq 0 \\
			5r^3\pi - 500 \geq 0
		\end{cases} \Longleftrightarrow \, \begin{cases}
		3r^3\pi - 500 + z^2 = 0 \\
		5r^3\pi - 500 - t^2 = 0
		\end{cases}
	\end{align*}

	De Lagrangefunctie wordt nu:
	\begin{align*}
	L(r,\lambda,\mu,z,t) &= 8r^2 + 1000r^{-1} + \lambda (3r^3\pi - 500 + z^2) + \mu (5r^3\pi - 500 - t^2) \\
						&= 8r^2 + 1000r^{-1} + 3\lambda r^3\pi - 500\lambda + \lambda z^2 + 5\mu r^3\pi - 500\mu - \mu t^2
	\end{align*}

	We zoeken de kritieke punten:
	\begin{align*}
		\begin{dcases}
			0 = \displaystyle\frac{\partial L}{\partial r} = 16r-1000r^{-2}+9\lambda r^2\pi + 15 \mu r^2 \pi \\
			0 = \displaystyle\frac{\partial L}{\partial \lambda} = 3r^3\pi - 500 + z^2 \\
			0 = \displaystyle\frac{\partial L}{\partial \mu} = 5r^3\pi - 500 - t^2 \\
			0 = \displaystyle\frac{\partial L}{\partial z} = 2\lambda z \\
			0 = \displaystyle\frac{\partial L}{\partial t} = -2\mu t
		\end{dcases}
	\end{align*}

	We onderscheiden 4 gevallen:
	\begin{itemize}
		\item \underline{\(t=0\) en \(z=0\)}: dit kan niet, aangezien uit vergelijking 1 en 2 dan volgt dat \(r=0\).
		\item \underline{\(\mu=0\) en \(z=0\)}: uit de tweede vergelijking volgt dat \[r=\sqrt[3]{\frac{500}{3\pi}}.\]
		In dit geval is \[t^2=5\pi\left(\frac{500}{3\pi}\right)-500>0,\] en dus is inderdaad \(t \in \R\).
		\item \underline{\(t=0\) en \(\lambda=0\)}: uit de tweede vergelijking volgt dat \[r=\sqrt[3]{\frac{100}{\pi}}.\]
		In dit geval is \[z^2=500-3\pi\left(\frac{100}{\pi}\right)>0,\] en dus is inderdaad \(z \in \R\).
		\newpage
		\item \underline{\(\mu=0\) en \(\lambda=0\)}:
		Uit de eerste vergelijking volgt dat (\(r\neq 0\)):
		\[16r-1000r^{-2} = 0 \Leftrightarrow 16r^3=1000 \Leftrightarrow r = \sqrt[3]{\frac{1000}{16}}=\sqrt[3]{\frac{125}{2}}\]
		We controleren of effectief geldt dat \(z,t \in \R\). Uit bovenstaande en uit vergelijking 2 volgt dat
		\[z^2=500-3\pi \left( \frac{125}{2} \right) < 0\]
		Als \(z\in\R\), kan \(z^2\) niet negatief zijn. Dit is dus geen mogelijk extremum.
	\end{itemize}

	We hebben 2 kandidaat-extrema, namelijk \(r=\sqrt[3]{\frac{500}{3\pi}}\) en \(r=\sqrt[3]{\frac{100}{\pi}}\). We controleren de functiewaarden voor beide stralen en kijken wat ons een minimum oplevert:
	\begin{itemize}
		\item \(\tilde{f}\left(\sqrt[3]{\frac{500}{3\pi}}\right) \approx 379.08 \)
		\item \(\tilde{f}\left(\sqrt[3]{\frac{100}{\pi}}\right) \approx 395.89 \)
	\end{itemize}

	\textbf{Besluit:}
	De producent verbruikt de minste hoeveelheid metaal voor \(r=\sqrt[3]{\frac{500}{3\pi}}\).

	\item[b)] Zou de producent veel meer winst hebben als hij geen esthetische eis zou opleggen? Gebruik je berekening uit a) om dit te illustreren.

	\underline{Oplossing:}

	Indien we niet kijken naar de esthetische voorwaarde, vinden we in het bovenstaande puntje a) voor het laatste geval (\(\mu=0\) en \(\lambda=0\)) wél een mogelijk extremum.
	We berekenen de functiewaarde voor deze straal:
	\[\tilde{f}\left(\sqrt[3]{\frac{125}{2}}\right) \approx 377.98\]
	De producent zou dus slechts ongeveer één vierkante centimeter minder metaal verbruiken.
\end{enumerate}

\newpage

\section*{Vraag 3}

Voor deze oefening gaan we omgekeerd te werk. Gegeven een oplossing van het homogene stelsel differentiaalvergelijkingen \(\vec{Y}'(t)=\textbf{A}\vec{Y}(t)\):
\[e^{-t}\left( \left[\begin{matrix}
	1\\
	-1
\end{matrix}\right]t + \left[ \begin{matrix}
1 \\
0
\end{matrix} \right] \right)\]

\begin{enumerate}
	\item[a)] Geef de matrix \(\textbf{A}\).

	\underline{Oplossing:}

	Het is duidelijk dat het hier om een \(2\times2\) stelsel gaat.
	Omdat er een veralgemeende eigenvector \(\vec{V}\) is voor de eigenwaarde \(\lambda=-1\), zal er voor deze eigenwaarde een defect zijn van \(\beta=1\). Dit is zo als \(m(-1)=1\) en \(d(-1)=2\).
	We kunnen \(\mathbf{A}\) schrijven als \(\mathbf{A}=\mathbf{PJP}^{-1}\) met
	\begin{align*}
		\mathbf{P} = \left( \begin{matrix}
			\vec{E} & \vec{V}
		\end{matrix} \right) = \left( \begin{matrix}
		1 & 1 \\
		-1 & 0
		\end{matrix} \right) & & \mathbf{J} = \left( \begin{matrix}
		-1 & 1 \\
		0 & -1
		\end{matrix} \right)
	\end{align*}
	Omdat nu geldt dat
	\[\textbf{P}^{-1}=\frac{1}{\text{det }\textbf{P}}\cdot\text{Adj }\textbf{P} = \left(\begin{matrix}
	0 & 1 \\
	-1 & 1
	\end{matrix}\right)^T=\left(\begin{matrix}
	0 & -1 \\
	1 & 1
	\end{matrix}\right)\]
	vinden we dat
	\[\mathbf{A}=\left(\begin{matrix}
		0 & 1 \\
		-1 & -2
	\end{matrix}\right)\]

	\item[b)] Vorm dit stelsel differentiaalvergelijkingen om tot een tweede orde differentiaalvergelijking van de vorm \(f(y'',y',y)=0\).

	\underline{Oplossing:}

	Gebruik de hulpfuncties
	\[	\begin{cases}
		y_1=y\\
		y_2=y'
	\end{cases}
	\]

	Dan volgt dat
	\begin{align*}
		&&\vec{Y}' &= \textbf{A}\vec{Y} & \\
		& \Leftrightarrow & \left( \begin{matrix}
			y_1' \\
			y_2'
		\end{matrix} \right) &= \left(\begin{matrix}
		0 & 1 \\
		-1 & -2
		\end{matrix}\right)\left( \begin{matrix}
		y_1 \\
		y_2
		\end{matrix} \right) & \\
		& \Leftrightarrow & \left( \begin{matrix}
			y' \\
			y''
		\end{matrix} \right) &= \left(\begin{matrix}
		0 & 1 \\
		-1 & -2
		\end{matrix}\right)\left( \begin{matrix}
		y \\
		y'
		\end{matrix} \right) &
	\end{align*}

	Als we dit uitschrijven, vinden we dat \[y''+2y'+y=0\]

	\item[c)] Los de niet-homogene vergelijking \(f(y'',y',y)=\frac{e^{-t}}{t}\) op. Als je b) niet gevonden had, mag je gebruiken dat \(f(y'',y',y)=y''+2y'+y\).

	\underline{Oplossing:}

	Uit de redenering in a) volgt dat
	\[\vec{Y}_h = c_1 e^{-t} \left[\begin{matrix}
		1\\
		-1
	\end{matrix}\right] + c_2 e^{-t}\left( \left[\begin{matrix}
		1\\
		-1
	\end{matrix}\right]t + \left[ \begin{matrix}
		1 \\
		0
	\end{matrix} \right] \right)\]

	Stel hiermee \(\textbf{Z}(t)\), \(\textbf{Z}^{-1}(t)\) en \(\vec{B}(t)\) op:
	\begin{align*}
		& \textbf{Z}(t) = e^{-t} \begin{bmatrix}
			1 & t+1 \\
			-1 & -t
		\end{bmatrix}, & & \textbf{Z}^{-1}(t) = e^{t} \begin{bmatrix}
		-t & -t-1 \\
		1 & 1
		\end{bmatrix}, & & \vec{B}(t) = \begin{bmatrix}
		0 \\
		\frac{e^{-t}}{t}
		\end{bmatrix}
	\end{align*}

	Bereken vervolgens \(\vec{Y}_p\) met de methode van variatie van constanten:
	\begin{align*}
		\vec{Y}_p	 &= \textbf{Z}(t) \int_{0}^{t} \textbf{Z}^{-1}(\xi)\vec{B}(\xi) \diff{\xi}\\
					&= \textbf{Z}(t) \int_{0}^{t} e^{\xi} \begin{bmatrix}
						-\xi & -\xi-1 \\
						1 & 1
					\end{bmatrix} \begin{bmatrix}
					0 \\
					\frac{e^{-\xi}}{\xi}
					\end{bmatrix} \diff{\xi}\\
					&= \textbf{Z}(t) \int_{0}^{t} \begin{bmatrix}
						-\frac{\xi+1}{\xi} \\
						\frac{1}{\xi}
					\end{bmatrix} \diff{\xi} \\
					&=e^{-t} \begin{bmatrix}
						1 & t+1 \\
						-1 & -t
					\end{bmatrix} \begin{bmatrix}
					-t-\ln{t}\\
					\ln{t} \end{bmatrix}\\
					&= e^{-t} \begin{bmatrix}
					\ln{t}-t\\
					t+\ln{t}-t\ln{t}
					\end{bmatrix}
	\end{align*}

	\underline{Alternatieve oplossing:}

	Los de lineaire differentiaalvergelijking \[y''+2y'+y=\frac{e^{-t}}{t}\] op zonder ze ontkoppelen in een stelsel differentiaalvergelijkingen.



\end{enumerate}







\end{document}
