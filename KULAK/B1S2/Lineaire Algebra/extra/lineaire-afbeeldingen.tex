\documentclass{article}
\usepackage[a4paper, margin={15mm, 15mm}]{geometry}

\usepackage{amsmath}
\usepackage{amssymb}
\usepackage{amsfonts} % for \mathbb

\newcommand{\R}{\mathbb{R}} % Real numbers
\newcommand{\C}{\mathbb{C}} % Complex numbers
\newcommand{\Q}{\mathbb{Q}}
\newcommand{\N}{\mathbb{N}}
\newcommand{\basis}[3]{#1^{#2}_{#3}}

\usepackage[dutch]{babel}
\usepackage{amsthm}

\setlength{\parindent}{0pt}

\title{Lineaire Afbeeldingen p164-165}
\author{Vincent Van Schependom}
\date{\today}

\begin{document}

	\maketitle
	
	\rule{1cm}{0.01mm}
	
	\textbf{Gegeven:}\\
	De surjectieve lineaire afbeelding \(L: V \to W\) en het voorbrengend deel \( \{ v_1,....,v_n \} \) voor \(V\)
	
	\textbf{Te bewijzen:}\\
	 \(W\) wordt voortgebracht door \( \{ L(v_1),...,L(v_n) \} \)
	
	\begin{proof}
		
		Neem \(w \in W\) willekeurig. \(L\) is surjectief, dus er geldt dat \(\exists v \in V: L(v)=w\). 
		
		Omdat \(V\) wordt voortgebracht door \( \{ v_1,....,v_n \} \), is \(v = \sum x_i v_i\) (\(x_i \in \R\)). 
		
		En dus volgt uit \( w = L(v) = L( \sum x_i v_i) =  \sum x_i L(v_i)\ \) dat elke \(w \in W\) een lineaire combinatie is van vectoren uit \( \{ L(v_1),...,L(v_n) \} \), wat betekent dat \(W\) wordt voortgebracht door \( \{ L(v_1),...,L(v_n) \} \).
	\end{proof}
	
	
	
	\rule{1cm}{0.01mm}
	
	\textbf{Gegeven:}\\
	De injectieve lineaire afbeelding \(L: V \to W\) en het vrij deel \( \{ v_1,....,v_n \} \) voor \(V\)
	
	\textbf{Te bewijzen:}\\
	\( \{ L(v_1),...,L(v_n) \} \) is vrij in \(W\)
	
	\begin{proof}
		
		Om te bewijzen dat \( \{ L(v_1),...,L(v_n) \} \) vrij is, nemen we een lineaire combinatie van de vectoren in deze verzameling en bewijzen we dat alle coëfficiënten in deze lineaire combinatie gelijk zijn aan \(0\).
		
		Stel dat \( \sum x_i L(v_i) = L( \sum x_i v_i) = 0 \). Omdat \(L\) injectief is en ook \( L(0) = 0 \), moet \( \sum x_i v_i   = 0\). Aangezien nu \( \{ v_1,....,v_n \} \) een vrij deel is, moeten de \(x_i\) allemaal gelijk zijn aan \(0\), wat wil zeggen dat ook \( \{ L(v_1),...,L(v_n) \} \) vrij is.
		
	\end{proof}
	
	
	
	\rule{1cm}{0.01mm}
	
	\textbf{Gegeven:}\\
	Het isomorfisme \(L: V \to W\) en de basis \( \{ v_1,....,v_n \} \) voor \(V\)
	
	\textbf{Te bewijzen:}\\
	\( \{ L(v_1),...,L(v_n) \} \) is een basis voor \(W\)
	
	\begin{proof}
		
		Omdat \( \{ v_1,....,v_n \} \) een basis is voor \(V\), is deze verzameling zowel vrij als voortbrengend. Anderzijds volgt uit het feit dat \(L\) een isomorfisme is, uiteraard dat \(L\) een lineaire afbeelding is.
		
		Uit voorgaande bewijzen volgt dan dat \( \{ L(v_1),...,L(v_n) \} \) voortbrengend voor \(W\) is en bovendien ook vrij, wat wil zeggen dat \( \{ L(v_1),...,L(v_n) \} \) een basis is voor \(W\).
		
	\end{proof}
	
\end{document}



