\documentclass{article}
\usepackage[a4paper, margin={15mm, 15mm}]{geometry}

\usepackage{amsmath}
\usepackage{amssymb}
\usepackage{amsfonts} % for \mathbb
\usepackage{amsthm}
\usepackage{tcolorbox}

\newcommand{\R}{\mathbb{R}} % Real numbers
\newcommand{\C}{\mathbb{C}} % Complex numbers
\newcommand{\Q}{\mathbb{Q}}
\newcommand{\N}{\mathbb{N}}

\theoremstyle{definition}
\newtheorem*{propositie}{Propositie}

\usepackage[dutch]{babel}

\setlength{\parindent}{0pt}

\title{Bewijs i.v.m. de voorwaarden van een directe som (algemener)}
\author{Vincent Van Schependom}
\date{\today}

\tcolorboxenvironment{propositie}{
	boxrule=1pt,
	boxsep=2pt,
	left=2pt,right=2pt,top=2pt,bottom=2pt,
	sharp corners,
	before skip=\topsep,
	after skip=\topsep,
}

\begin{document}
	
	\maketitle
	
	\begin{propositie}
		Gegeven is een vectorruimte \((\R,V,+)\).
		Zij \(U_1,U_2,...,U_k\) deelruimten van \(V\). Dan is \[W=\oplus_{i=1}^k U_i\] als en slechts als
		\begin{enumerate}
			\item[(a)] \(W=\sum_{i=1}^k U_i\)
			\item[(b)] voor alle \(i=1,...,k\) geldt dat \[U_i \cap (U_1+...+U_{i-1}+U_{i+1}+...+U_k)=\{0\} \]
		\end{enumerate}
	\end{propositie}

	\begin{proof}
		
		Veronderstel dat \(W=\oplus_{i=1}^k U_i\). Voor een willekeurige vector \(w \in W\) geldt dan dat er unieke vectoren
		\(u_1 \in U_1, u_2 \in U_2, ..., u_k \in U_k\) bestaan, zó dat \(w=\sum_{i=1}^{k}u_i\). Dus is zeker \(W = \sum_{i=1}^k U_i\). \\
		
		We beweren dat \(U_i \cap (U_1+...+U_{i-1}+U_{i+1}+...+U_k)=\{0\}\).
		Veronderstel immers dat \[0\neq v \in U_i \cap (U_1+...+U_{i-1}+U_{i+1}+...+U_k).\]
		Dan kunnen we die vector \(v\) schrijven als \[v=v+0=0+v,\] waarbij de eerste
		keer \(v\) als vector van \(U_i\) en de tweede keer \(v\) als vector van \((U_1+...+U_{i-1}+U_{i+1}+...+U_k)\) opgevat wordt. Bijgevolg zou de somruimte \(U_i+(U_1+...+U_{i-1}+U_{i+1}+...+U_k) = \sum_{i=1}^k U_i\) geen directe som zijn, wat een tegenspraak levert met het gegeven. De doorsnede \(U_i \cap (U_1+...+U_{i-1}+U_{i+1}+...+U_k)\) moet dus wel beperkt zijn tot \(\{0\}\). \\
		
		Veronderstel nu omgekeerd dat \(W=\sum_{i=1}^k U_i\) en dat voor alle \(i=1,...,k\) geldt dat \[U_i \cap (U_1+...+U_{i-1}+U_{i+1}+...+U_k)=\{0\}. \]
		We tonen aan dat de som \(W=\sum_{i=1}^k U_i = U_i+(U_1+...+U_{i-1}+U_{i+1}+...+U_k)\) een directe som is. Veronderstel dat er een vector \(w \in W\) is die op meerdere wijzen als som van een vector uit \(U_i\) en \((U_1+...+U_{i-1}+U_{i+1}+...+U_k)\) kan geschreven worden (met \(i \in \{1,...,k\} \)). Veronderstel met andere woorden dat \[w=u_i+u_j=u_i'+u_j',\] met
		\(u_i,u_i' \in U_i\) en \(u_j,u_j' \in (U_1+...+U_{i-1}+U_{i+1}+...+U_k)\). Dan volgt dat \[u_i-u_i'=u_j'-u_j \in U_i \cap (U_1+...+U_{i-1}+U_{i+1}+...+U_k) = \{0\}.\] Bijgevolg moet \(u_i=u_i'\) en \(u_j=u_j'\).
		
	\end{proof}
	
	
\end{document}



