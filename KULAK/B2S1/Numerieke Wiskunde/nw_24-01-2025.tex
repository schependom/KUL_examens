\documentclass[kulak]{kulakarticle}

\usepackage{amsmath}
\usepackage{amssymb}
\usepackage{amsfonts}
\usepackage{amsthm}
\usepackage{tcolorbox}
\usepackage{mathtools}
\usepackage{siunitx}
\usepackage{cancel}
\usepackage{mathtools}
\usepackage{enumitem}

\DeclareUnicodeCharacter{2212}{I~AM~HERE!!!!}

\let\epsilon\varepsilon

\newcommand{\R}{\mathbb{R}} % Real numbers
\newcommand{\C}{\mathbb{C}} % Complex numbers
\newcommand{\Q}{\mathbb{Q}}
\newcommand{\N}{\mathbb{N}}
\DeclareMathOperator{\powerset}{\mathcal{P}}

\DeclarePairedDelimiter\abs{\lvert}{\rvert}
\usepackage{array}
\usepackage{multirow}

\sisetup{output-decimal-marker={.}}
\sisetup{per-mode=symbol}
\sisetup{per-symbol=/}
\sisetup{group-digits=integer}

\newcommand{\rood[1]}{\color{red}#1\color{black}}

\usepackage[dutch]{babel}
\usepackage{hyperref}

%\setlength{\parindent}{0pt}
% d voor dx
\newcommand*\diff{\mathop{}\!\mathrm{d}}

\usepackage{parskip}

\title{Examen Numerieke Wiskunde}
\author{ViS}
\date{24 januari 2025}
\address{
	\textbf{Numerieke Wiskunde}\\
	X0A43A (I2, N2, W2)\\
	Prof. Koen Van Den Abeele\\
	Drs. Marie Cloet}

\begin{document}

	\maketitle

	Het examen bestaat uit 3 delen:

	\begin{table}[h!]
		\renewcommand{\arraystretch}{1.3}
		\begin{tabular}{r l}
			\hline
			                               \multicolumn{2}{c}{\textbf{Theorie}}                                \\ \hline
			                  Duur & 60 minuten                                                                \\
			              Tijdslot & \textit{tijdslot individueel per student}                                 \\
			         Voorbereiding & op rij 1 en rij 3, vooraan in het lokaal                                  \\
			Mondelinge verdediging & Bij prof. Van Den Abeele in A341                                          \\
			          Hulpmiddelen & Eigen schrijfgerief                                                       \\
			                       & Formularium is voorzien                                                   \\
			                 Score & 8 punten                                                                  \\ \hline
			                        \multicolumn{2}{c}{\textbf{Bespreking practicum}}                          \\ \hline
			                  Duur & 10 minuten                                                                \\
			              Tijdslot & Aansluitend op mondelinge verdediging theorie                             \\
			         Voorbereiding & n.v.t.                                                                    \\
			Mondelinge verdediging & Bij de assistent in A352                                                  \\
			          Hulpmiddelen & Het verslag is voorzien bij Marie                                         \\
			                 Score & 4 punten (volledig practicum)                                             \\ \hline
			                             \multicolumn{2}{c}{\textbf{Oefeningen}}                               \\ \hline
			                  Duur & 160 minuten                                                               \\
			              Tijdslot & \textit{elk moment voor en na het theoriegedeelte}                        \\
			Mondelinge verdediging & n.v.t.                                                                    \\
			          Hulpmiddelen & Handboek, slides, eigen notities, oefeningenbundel, grafisch rekenmachine \\
			                       & \textbf{Geen opgeloste oefeningen}                                        \\
			                 Score & 8 punten \\ \hline
		\end{tabular}
	\end{table}

	Nog enkele algemene instructies:
	\begin{itemize}
		\item Het consulteren van hulpmiddelen die niet toegelaten zijn bij de verschillende onderdelen wordt aanzien als examenfraude.
		\item Je geeft telkens alles (vragen, antwoorden, klad) af.
		\item Zorg dat op kladpapier duidelijk de vermelding KLAD staat.
		\item ZORG DAT JE NAAM OP ELK BLAD STAAT.
		\item Werk gestructureerd en netjes.
	\end{itemize}

	\flushright

	\textbf{Veel succes!}

	\flushleft

	\newpage

	\section*{Theorie}

	\subsection*{Vraag 1}

	\begin{enumerate}
		\item Bespreek de conditie van het probleem ``zoek het nulpunt van een niet-lineaire functie''. Laat dit ook zien op tekeningen.
		\item Geef 3 methodes voor het vinden van het nulpunt van een niet-lineaire functie, waaronder 1 meerstapsmethode en 1 niet-stationaire methode. Bespreek voor elk van de methoden het principe, de convergentiesnelheid en de voor- en nadelen.
		\item Wat verstaat men onder de begrippen convergentiefactor en orde van convergentie?
		\item Bespreek de voorwaarde(n) voor convergentie bij substitutiemethodes waarbij \(x^{(k+1)}=F\left(x^{(k)}\right)\). Toon dit ook visueel aan.
		\item Leg uit waarom de methode van Newton-Raphson voor enkelvoudige nulpunten steeds convergeert.
		\item Toon aan dat de convergentiefactor een invloed heeft op de stabiliteit van een convergente substitutiemethode.
		\item Wat is de numerieke betekenis van de convergentiefactor?
		\item Verklaar waarom de methode van Halley sneller convergeert dan de methode van Newton-Raphson.
	\end{enumerate}

	\subsection*{Vraag 2}

	\textit{Geef kort het antwoord op onderstaande vraag; het is niet nodig om dit heel uitgebreid uit te schrijven.}
	Verklaar waarom een gedeelde differentie onafhankelijk is van de volgorde van de gebruikte punten.

	\newpage

	\section*{Oefeningen}

	\subsection*{Vraag 1}

	Beschouw onderstaand stelsel van lineaire vergelijkingen.
	\begin{align*}
		\begin{cases}
			-3x + y = 5\\
			-6x + 4y - z = 3\\
			-6x + 8y - 6z = 1
		\end{cases}
	\end{align*}

	\begin{enumerate}

		\item Stel dat we handmatig de \(LU\)-ontbinding van dit stelsel zouden willen bepalen.
		\begin{enumerate}
			\item Is het noodzakelijk om een pivotering toe te passen? Verklaar waarom wel/niet.
			\item Heeft het zin om optimale pivotering toe te passen? Verklaar waarom wel/niet.
			\item Bereken de \(LU\)-ontbinding van de systeemmatrix, indien nodig met optimale pivotering.
		\end{enumerate}

		\item Stel dat we iteratief de oplossing zouden willen berekenen aan de hand van de methode van Jacobi of de methode van Gauss-Seidel.\\\textit{Hint: voor deze vraag kan je gerust je GRM gebruiken.}
		\begin{enumerate}
			\item Speelt de volgorde van de vergelijkingen een rol? Leg duidelijk uit waarom wel/niet.
			\item Ben je zeker dat de methoden zullen convergeren?
			\item Reken het resultaat van beide methoden uit na 2 iteratiestappen, gegeven dat \(x_0 = [1, \, 1 \, 1]^\top\). Is de uitkomst zoals je verwacht, als je weet dat de exacte oplossing gelijk is aan \([-1 \,, 30 \,, -17]^\top /8\)?
		\end{enumerate}

	\end{enumerate}

	\newpage

	\subsection*{Vraag 2}

	In het practicum over numerieke integratie werden onder andere onderstaande twee (correcte) foutenplots teruggevonden voor de integraal \[I_5 = \int_{0}^{10} 13(x-x^2)\exp\left(-\frac{3x}{2}\right) \diff x,\]waarop de convergentie van de middelpuntsregel te zien is:

	\begin{table}[h!]
		\centering
		\begin{tabular}{|c|}
			\hline \\
			Loglogplot van de middelpuntsregel \\ met (dalend) lineair verband\\
			en met afvlakking op het eind. \\ \\\hline
		\end{tabular}
		\hspace{1cm}
		\begin{tabular}{|c|}
		\hline\\
		Semilog foutenplot van de \\
		middelpuntsregel, de trapeziumregel\\
		en de regel van Simpson.  \\ \\\hline
		\end{tabular}
	\end{table}

	\begin{enumerate}
		\item Op welke grafiek kan je het makkelijkst de convergentie-orde van de middelpuntsregel bepalen? Tijdens de oefenzitting gebruikten we steevast \texttt{semilogy} voor het maken van foutenplots. Is dat bij deze beste grafiek ook het geval? Verklaar waarom.

		\item Gebruik de grafiek om de convergentie-orde van de middelpuntsregel te schatten en vergelijk die met de theoretische convergentie-orde.\\
		\textit{Hint: je kan gebruik maken van het feit dat \(f(x)=13(x-x^2)\exp\left(-\frac{3x}{2}\right)\) volgende afgeleiden heeft:}
		\begin{align*}
			f'(x) &= 13\left(1-\frac{7}{2}x+\frac{3}{2}x^2\right) \exp\left(-\frac{3x}{2}\right)\\
			f''(x) &= 13\left(-5+\frac{33}{4}x-\frac{9}{4}x^2\right) \exp\left(-\frac{3x}{2}\right)
		\end{align*}

		\item Op figuur (a) zien we dat de fout stagneert na \(10^{8}\) functie-evaluaties. Hoe kan je dit verklaren?

	\end{enumerate}

	\newpage

	\subsection*{Vraag 3}

	\begin{enumerate}

		\item Geef alle veeltermen met minimale graad die voldoen aan \[p(-1)=p(0)=p(-1)=a, \quad p'(0)=1\] door gebruik te maken van Hermite interpolatie.

		\item Bepaal in detail de conditie tegenover zowel de absolute als de relatieve fouten wanneer we de gevonden veeltermen evalueren in \(a\).

		\item Geef twee methodes om de veelterm te evalueren en bespreek kort welke het meest stabiel is. Een uitgebreide foutenanalyse is dus niet nodig.

	\end{enumerate}

\end{document}



