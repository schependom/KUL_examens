\documentclass{kuburgiearticle}

\usepackage{amsmath}
\usepackage{amssymb}
\usepackage{amsfonts}
\usepackage{amsthm}
\usepackage{tcolorbox}
\usepackage{mathtools}
\usepackage{siunitx}
\usepackage{cancel}
\usepackage{mathtools}

\DeclareUnicodeCharacter{2212}{I~AM~HERE!!!!}

\let\epsilon\varepsilon

\newcommand{\R}{\mathbb{R}} % Real numbers
\newcommand{\C}{\mathbb{C}} % Complex numbers
\newcommand{\Q}{\mathbb{Q}}
\newcommand{\N}{\mathbb{N}}

\DeclareSIUnit\biti{bit_i}
\DeclareSIUnit\symbool{symbool}
\DeclareSIUnit\symbolen{symbolen}
\DeclareSIUnit\combinatie{combinatie}
\DeclareSIUnit\baud{baud}
\newcommand{\sam}{\text{sam}}

\DeclarePairedDelimiter\abs{\lvert}{\rvert}
\usepackage{array}
\usepackage{multirow}

\sisetup{output-decimal-marker={.}}
\sisetup{per-mode=symbol}
\sisetup{per-symbol=/}
\sisetup{group-digits=integer}

\newcommand{\rood[1]}{\color{red}#1\color{black}}

\usepackage[dutch]{babel}
\usepackage{hyperref}

%\setlength{\parindent}{0pt}
% d voor dx
\newcommand*\diff{\mathop{}\!\mathrm{d}}

\usepackage{parskip}

\title{Examen Informatieoverdracht en -verwerking}
\author{Vincent Van Schependom}
\date{9 januari 2025}
\address{
	\textbf{Informatieoverdracht en -verwerking}\\
	Prof. Lieven De Lathauwer \& Ben Hermans}

\begin{document}

	\maketitle

	\section*{Vraag 1}

	Een koning maakt een random walk op een 3x3 schaakbord met posities 1 tot en met 9. We kunnen dit modelleren door de hoeken te benoemen met \(H\), de posities aan de zijden met \(R\), en het midden met \(C\).

	\renewcommand{\arraystretch}{1.5}
	\begin{table}[h!]
		\centering
		\begin{tabular}{|c|c|c|}
			\hline
			1 & 2 & 3 \\
			\hline
			4 & 5 & 6 \\
			\hline
			7 & 8 & 9 \\
			\hline
		\end{tabular}
		\hspace{2cm}
		\begin{tabular}{|c|c|c|}
			\hline
			\(H\) & \(R\) & \(H\) \\
			\hline
			\(R\) & \(C\) & \(R\) \\
			\hline
			\(H\) & \(R\) & \(H\) \\
			\hline
		\end{tabular}
	\end{table}

	\begin{enumerate}
		\item Bereken de gemiddelde hoeveelheid informatie voor een voorstelling \(\{R,H,C\}\) na heel veel stappen.
		\item We duiden nu elke positie aan m.b.v. diens getal. Wat zijn de kansen voor een voorstelling \(\{1,2,...,9\}\)? Bereken ook hiervoor de gemiddelde hoeveelheid informatie.
		\item Als we de positie van de koning doorsturen door eerst de rij en vervolgens de kolom door te geven, bestaat er dan een afhankelijkheid tussen de rijen en kolommen? Leg uit.
		\item Maak een Huffman-codering voor een voorstelling \(\{1,2,...,9\}\). Bereken de gemiddelde codewoordlengte, de efficiëntie en de compressieverhouding van je codering.
	\end{enumerate}

	\section*{Vraag 2}

	Gegeven twee figuren met daarop het frequentiespectrum van twee filters; de linkerfiguur is een gewone plot en de rechterfiguur is een semilog plot. De ene filter is in het blauw getekend, de andere in het rood. Er geldt dat \(M=51\) en \(f_\text{sam}=\SI{100}{\mega\hertz}\). De cut-off frequentie voor de blauwe figuur is 1. De stopfrequentie van de blauwe filter is aangeduid op de rechterfiguur.

	\begin{enumerate}
		\item Welke windows worden hier gebruikt? Leg uit.
		\item Wat is de vertraging?
		\item Bepaal de doorlaatfrequentie, stopfrequentie en cut-off frequentie van de blauwe filter in Hertz.
		\item Bereken de breedte van de transitieband van de blauwe filter. Komt die overeen met de uitdrukking het eerder bepaalde window?
		\item Hoe zou een filter met een Hann window eruit zien? Teken deze filter op beide figuren en motiveer je schets.
	\end{enumerate}

	\section*{Vraag 3}

	Wanneer is de performantie van een \((n,k)\)-blokcode het hoogst? Hiermee bedoelen we: wanneer is het foutdetectievermogen maximaal? We beperken ons tot \(k=3\).

	\begin{enumerate}
		\item Geef de (\(4,3)\)-blokcode met de beste performantie en bepaal het foutdetecterend vermogen.
		\item Bepaal alle codewoorden van de zonet bepaalde blokcode en geef ook het foutcorrectievermogen.
		\item Bepaal de efficiëntie. Als je een extra kolom toevoegt, wat gebeurt er dan met de efficiëntie en de performantie van je blokcode?
	\end{enumerate}

	\section*{Vraag 4}

	Een signaal wordt verstuurd in basisband aan een transmissiedebiet van \SI{60}{\kilo\bit\per\second}. Hiervoor worden \(M\) verschillende golfvormen gebruikt met factor \(\alpha = 0.75\). Verder geldt dat \(E_b/N_0=100\). We willen een foutkans die minder dan \(10^{-6}\) bedraagt.

	\begin{enumerate}
		\item Hoeveel golfvormen kunnen we maximaal versturen?
		\item Wat is de bandbreedte van dit signaal?
		\item We passen FDM toe van 100 tot \SI{180}{\kilo\hertz} met enkel-zijband modulatie, waarbij enkel de hoogste zijband wordt behouden. Hoeveel signalen kunnen we maximaal multiplexen? Stel dat we maximale spreiding willen -- ook bij \SI{100}{\kilo\hertz}, maar niet bij \SI{180}{\kilo\hertz}. Bepaal dan de draaggolffrequenties en teken het spectrum.
	\end{enumerate}

	\section*{Vraag 5}

	We ontwerpen een (kinder)slot met 3 cijfers. De invoer voor het slot zijn combinaties \(A_1 A_0\), die een binaire voorstelling zijn van de cijfers die werden gegeven. Bijvoorbeeld: \(01\) wil zeggen dat het cijfer 1 werd ingevoerd; \(11\) wil zeggen dat het cijfer 3 werd ingevoerd. De uitvoer van het slot is \(Y\) en is gelijk aan 0 wanneer een verkeerde combinatie werd ingevoerd en 1 wanneer de juiste combinatie -- namelijk 3, gevolgd door 0, gevolgd door 2 -- werd ingevuld. Wanneer een fout cijfer wordt gegeven, verwerpt het slot de volledige sequentie niet, maar blijft het in dezelfde toestand. Wanneer de correcte combinatie werd ingevuld, moet het slot automatisch terug naar de begintoestand gaan in de volgende klokcyclus.

	\begin{enumerate}
		\item Teken een Moore toestandsdiagram van dit slot.
		\item Codeer de toestanden aan de hand van straight-forward codering en Gray codering. Wat zijn de voordelen van elke codering? Tip: wat stelt de straight-forward codering van een toestand voor?
		\item Stel de waarheidstabel op voor de ingang, huidige toestand, volgende toestand en de output.
		\item Bepaal de SOP uitdrukkingen voor de volgende toestand en de output.
		\item Stel Karnaugh kaarten op voor de volgende toestand en de output. Bepaal minimale uitdrukkingen in Booleaanse logica met AND, OR en NOT.
		\item Teken de schakeling op een overzichtelijke manier op poortniveau.
		\item Voer in een andere kleur (of in het potlood) technologiemapping (CMOS) uit.
	\end{enumerate}

\end{document}



